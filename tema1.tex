%%%%%%%%%%%%%%%%%%%%%%%%%%%%%%%%%%%%%%%%%
% Journal Article
% LaTeX Template
% Version 1.4 (15/5/16)
%
% This template has been downloaded from:
% http://www.LaTeXTemplates.com
%
% Original author:
% Frits Wenneker (http://www.howtotex.com) with extensive modifications by
% Vel (vel@LaTeXTemplates.com)
%
% License:
% CC BY-NC-SA 3.0 (http://creativecommons.org/licenses/by-nc-sa/3.0/)
%
%%%%%%%%%%%%%%%%%%%%%%%%%%%%%%%%%%%%%%%%%

%----------------------------------------------------------------------------------------
%	PACKAGES AND OTHER DOCUMENT CONFIGURATIONS
%----------------------------------------------------------------------------------------

\documentclass[twoside,twocolumn]{article}

\usepackage{blindtext} % Package to generate dummy text throughout this template 

\usepackage[sc]{mathpazo} % Use the Palatino font
\usepackage[T1]{fontenc} % Use 8-bit encoding that has 256 glyphs
\linespread{1.05} % Line spacing - Palatino needs more space between lines
\usepackage{microtype} % Slightly tweak font spacing for aesthetics

\usepackage[english]{babel} % Language hyphenation and typographical rules

\usepackage[hmarginratio=1:1,top=32mm,columnsep=20pt]{geometry} % Document margins
\usepackage[hang, small,labelfont=bf,up,textfont=it,up]{caption} % Custom captions under/above floats in tables or figures
\usepackage{booktabs} % Horizontal rules in tables

\usepackage{lettrine} % The lettrine is the first enlarged letter at the beginning of the text

\usepackage{enumitem} % Customized lists
\setlist[itemize]{noitemsep} % Make itemize lists more compact

\usepackage{abstract} % Allows abstract customization
\renewcommand{\abstractnamefont}{\normalfont\bfseries} % Set the "Abstract" text to bold
\renewcommand{\abstracttextfont}{\normalfont\small\itshape} % Set the abstract itself to small italic text

\usepackage{titlesec} % Allows customization of titles
\renewcommand\thesection{\Roman{section}} % Roman numerals for the sections
\renewcommand\thesubsection{\roman{subsection}} % roman numerals for subsections
\titleformat{\section}[block]{\large\scshape\centering}{\thesection.}{1em}{} % Change the look of the section titles
\titleformat{\subsection}[block]{\large}{\thesubsection.}{1em}{} % Change the look of the section titles

\usepackage{fancyhdr} % Headers and footers
\pagestyle{fancy} % All pages have headers and footers
\fancyhead{} % Blank out the default header
\fancyfoot{} % Blank out the default footer
\fancyhead[C]{Running title $\bullet$ May 2016 $\bullet$ Vol. XXI, No. 1} % Custom header text
\fancyfoot[RO,LE]{\thepage} % Custom footer text

\usepackage{titling} % Customizing the title section

\usepackage{hyperref} % For hyperlinks in the PDF

%----------------------------------------------------------------------------------------
%	TITLE SECTION
%----------------------------------------------------------------------------------------

\setlength{\droptitle}{-4\baselineskip} % Move the title up

\pretitle{\begin{center}\Huge\bfseries} % Article title formatting
\posttitle{\end{center}} % Article title closing formatting
\title{Un algoritmo de dos fases para el reconocimiento de las actividades humanas en el contexto de la Industria 4. 0 y procesos} % Article title
\author{%
\textsc{Borja Bordel$^1$ , Ramón Alcarria$^1$, Diego Sánchez-de-Rivera$^1$} \\[1ex] % Your name
\normalsize $^1$Universidad Politécnica de Madrid,
Madrid, España \\ % Your institution
\normalsize \href{mailto:bbordel@dit.upm.es, ramon.alcarria@upm.es, diegosanchez@dit.upm.es}{bbordel@dit.upm.es, ramon.alcarria@upm.es, diegosanchez@dit.upm.es} % Your email address
%\and % Uncomment if 2 authors are required, duplicate these 4 lines if more
%\textsc{Jane Smith}\thanks{Corresponding author} \\[1ex] % Second author's name
%\normalsize University of Utah \\ % Second author's institution
%\normalsize \href{mailto:jane@smith.com}{jane@smith.com} % Second author's email address
}
\date{\today} % Leave empty to omit a date
\renewcommand{\maketitlehookd}{%
\begin{abstract}
\noindent \blindtext % Dummy abstract text - replace \blindtext with your abstract text
\end{abstract}
}

%----------------------------------------------------------------------------------------

\begin{document}

% Print the title
\maketitle

%----------------------------------------------------------------------------------------
%	ARTICLE CONTENTS
%----------------------------------------------------------------------------------------

\section{Introduction}

\lettrine[nindent=0em,lines=3]{L} orem ipsum dolor sit amet, consectetur adipiscing elit.
\blindtext % Dummy text

\blindtext % Dummy text

%------------------------------------------------

\section{Methods}

Maecenas sed ultricies felis. Sed imperdiet dictum arcu a egestas. 
\begin{itemize}
\item Donec dolor arcu, rutrum id molestie in, viverra sed diam
\item Curabitur feugiat
\item turpis sed auctor facilisis
\item arcu eros accumsan lorem, at posuere mi diam sit amet tortor
\item Fusce fermentum, mi sit amet euismod rutrum
\item sem lorem molestie diam, iaculis aliquet sapien tortor non nisi
\item Pellentesque bibendum pretium aliquet
\end{itemize}
\blindtext % Dummy text

Text requiring further explanation\footnote{Example footnote}.

%------------------------------------------------

\section{Results}

\begin{table}
\caption{Example table}
\centering
\begin{tabular}{llr}
\toprule
\multicolumn{2}{c}{Name} \\
\cmidrule(r){1-2}
First name & Last Name & Grade \\
\midrule
John & Doe & $7.5$ \\
Richard & Miles & $2$ \\
\bottomrule
\end{tabular}
\end{table}

\blindtext % Dummy text

\begin{equation}
\label{eq:emc}
e = mc^2
\end{equation}

\blindtext % Dummy text

%------------------------------------------------

\section{Discussion}

\subsection{Subsection One}

A statement requiring citation \cite{Figueredo:2009dg}.
\blindtext % Dummy text

\subsection{Subsection Two}

\blindtext % Dummy text

%----------------------------------------------------------------------------------------
%	REFERENCE LIST
%----------------------------------------------------------------------------------------

\begin{thebibliography}{99} % Bibliography - this is intentionally simple in this template

\bibitem[}
1. Bordel, B., Alcarria, R., Sánchez-de-Rivera, D., & Robles, T. (2017, noviembre). Proteger los sistemas de la industria 4.0 contra los efectos maliciosos de los ataques ciberfísicos. En Conferencia Internacional sobre Computación Ubicua e Inteligencia Ambiental (págs. 161 a 171). Saltar, Cham.

2. Bordel, B., Alcarria, R., Robles, T., & Martín, D. (2017). Sistemas ciberfísicos: Extender la detección omnipresente de la teoría del control al Internet de las Cosas. Computación generalizada y móvil, 40, 156-184.

3. Neff, W. (2017). Trabajo y comportamiento humano. Routledge.

4. Bordel, B., Alcarria, R., Martín, D., Robles, T., & de Rivera, D. S. (2017). Auto-
configuración en sistemas ciberfísicos humanizados. Revista de Inteligencia Ambiental y
Computación Humanizada, 8 (4), 485-496.

5. Bordel, B., de Rivera, D. S., Sánchez-Picot, Á., & Robles, T. (2016). Procesos físicos
control en los sistemas basados en la industria 4.0: un enfoque en los sistemas ciberfísicos. En Ubiquitous
Computación e Inteligencia Ambiental (págs. 257-262). Saltar, Cham.

6. Pal, S. K., & Wang, P. P. (2017). Algoritmos genéticos para el reconocimiento de patrones. La prensa del CRC.

7. Müller, M. (2007). Deformación dinámica del tiempo. Recuperación de información para música y movimiento,
69 y 84.

8. Eddy, S. R. (1996). Modelos ocultos de Markov. Dictamen actual en biología estructural, 6 (3),
361-365.

9. Kim, E., Helal, S., y Cook, D. (2010). Reconocimiento de actividad humana y descubrimiento de patrones.
IEEE Pervasive Computing/IEEE Computer Society [y] IEEE Communications
Sociedad, 9 1), 48.

10. Li, Z., Wei, Z., Yue, Y., Wang, H., Jia, W., Burke, L. E., ... & Sun, M. (2015). Un Modelo de markov oculto adaptable para el reconocimiento de actividad basado en un multi-sensor portátil
dispositivo. Revista de sistemas médicos, 39 (5), 57.
\newblock Assortative pairing and life history strategy - a cross-cultural
  study.
\newblock {\em Human Nature}, 20:317--330.
 
\end{thebibliography}

%----------------------------------------------------------------------------------------

\end{document}
